\documentclass[11pt]{article}

\usepackage[top=0.5in, bottom=0.5in, left=0.5in, right=0.5in]{geometry}
\usepackage{authblk}
\usepackage{hyperref}
\usepackage[utf8]{inputenc}
\usepackage{amsmath}
\usepackage{amsfonts}
\usepackage{amssymb}
\usepackage{siunitx}
\usepackage{graphicx}
\usepackage{subcaption}
\usepackage{float}
\usepackage[nottoc,numbib]{tocbibind}
\usepackage{biblatex}

\bibliography{references.bib}

\newcommand{\email}[1]{\texttt{\href{mailto:#1}{#1}}}

\title{CMSC 411 Project Documentation}
\author{Robert Rose, Alex Flaerty and Mina Beshai}

\makeatletter
\let\inserttitle\@title
\let\insertauthor\@author
\makeatother

\begin{document}

\begin{center}
  \LARGE{\inserttitle}

  \Large{\insertauthor}
\end{center}

\section{Approach}

In the beginning we were hoping to implement the algorithms entirely on our own, but unfortunately
the resources out there on CORDIC, especially beyond simply sine and cosine, are rather sparse.
Although the original CORDIC paper is somewhat helpful, we didn't initially go looking for
it.\cite{volder1959cordic}\\

When we first went to research sine and cosine approximation, we mostly got resources detailing Taylor
series and approximation using multiplication.\cite{coranac} It wasn't until class on Thursday when
Professor Cain sent out code he found and talked about the project in class that we knew what to
look for.\cite{cainemail} After some digging on GitHub, Mina found an old project from last year that
we could tear apart and re-document.\cite{oldproj} We then went through and improved the documentation
for that project and then looked for any areas of improvement. We did actually manage to find one area
where the code could potentially be improved. In addressing the lookup table, we can potentially shave
off one instruction by using ARM's register offset addressing instead of adding to the address register
and then referencing it.\cite{registeroffsetmanual}\\

We also attempted the extra credit, since partial credit is awarded. One of the sources we had provided
C code for a $log()$ estimator but we chose not to implement it in ARM because we weren't sure if $log()$
was available for extra credit, only $ln()$ is listed on the handout.\cite{efunclog} After submitting the
main project as well we also tried to write $sqrt()$ in ARM as well and succeeded, although with using
multiply in order to simplify the algorithm. We found C code online and then used the sine-cosine ARM code
as a base in order to write it in ARM.\cite{sqrtcordic}

\section{Results}

In order to get the CPI and other stats, we first needed to run ARMSim multiple times in order to get
the average number of instructions per second from the console output. We borrowed this portion of the
procedure from a previous group's project that we found on GitHub.\cite{oldreport} For the sine and
cosine script we ran eight trials but one of the trials resulted in an extremely large instructions
per second, so it was discarded as an outlier. When we reran that run with the same inputs, the result
was equally as large. We're not sure if this is an error in the program or simply something that happens
for some values.\\

\begin{equation}
    IPS_{sincos} = \frac{4739 + 4839 + 4675 + 4829 + 4926 + 4387 + 4939}{7} = 4762
\end{equation}

\begin{equation}
    INS_{sincos} = \frac{244 + 242 + 242 + 242 + 246 + 241 + 244}{7} = 243
\end{equation}

Unfortunately, as we're working on the report now we realized that the previous report we found wasn't
actually correct.\cite{oldreport} We don't actually know how we're supposed to calculate the CPI without
manually stepping through the program, something not really possible since it has hundreds of instructions
total. After talking it over with our team and not hearing back from Professor Cain in time we decided to
search for any ARMSim plugins to help us.\\

Our contingency plan for if we cannot find an ARMSim plugin or option to calculate instruction shares for
us is to take the average CPI of the instructions we used and multiplying it by the average count of number of Instructions returned by ARMSim.

\section{Input/Output}

Here we've detailed the input and output of our program. For sine and cosine, we used degrees and for the
hyperbolic equations we used radians as our results were closer to the ones the professor outlined in the
email he sent out.\cite{profemailresults}

\begin{table}[h]
 \centering
 \caption{Sine and Cosine Input and Output}
 \begin{tabular}{||c c c c||}
 \hline
 Degrees & Hex & Sine & Cosine \\ [0.5ex]
 \hline\hline
 14.3239    & 000E 52EB & 0000 F803 & 0000 3F70 \\
 \hline
 15         & 000F 0000 & 0000 F744 & 0000 4253 \\
 \hline
 28.6479    & 001C A5DC & 0000 E09E & 0000 7AD2 \\
 \hline
 42.9718    & 002A F8C7 & 0000 BB4E & 0000 AE84 \\
 \hline
 45.5       & 002D 8000 & 0000 B361 & 0000 B6A5 \\
 \hline
 57.2958    & 0039 4BB9 & 0000 8A48 & 0000 D770 \\
 \hline
 66         & 0042 0000 & 0000 6809 & 0000 E9E8 \\
 \hline
 89.125     & 0059 2000 & 0000 03F2 & 0000 FFF6 \\
 \hline
\end{tabular}
\end{table}

\begin{table}[h]
 \centering
 \caption{Sinh and Cosh Input and Output}
 \begin{tabular}{||c c c c||}
 \hline
 Radians & Hex & Sinh & Cosh \\ [0.5ex]
 \hline\hline
 $\frac{1}{4}$  & 0000 4000 & 0000 409C & 0001 48B4 \\
 \hline
 0.2618         & 0000 4305 & 0000 4357 & 0001 4C9B \\
 \hline
 $\frac{1}{2}$  & 0000 8000 & 0000 8533 & 0001 A60F \\
 \hline
 $\frac{3}{4}$  & 0000 C000 & 0000 D26D & 0002 1DF6 \\
 \hline
 0.7941         & 0000 CB4A & 0000 E14D & 0002 3666 \\
 \hline
 1              & 0001 0000 & 0001 2CB2 & 0002 B7E4 \\
 \hline
 1.1519         & 0001 26E2 & 0001 6C7D & 0003 2A09 \\
 \hline
 1.5533         & 0001 8DA5 & 0002 41EE & 0004 BA1D \\
 \hline
\end{tabular}
\end{table}


% \begin{figure}[h]
% \caption{K-Means with Five Clusters}
% \includegraphics[width=0.6\textwidth]{clusters.png}
% \centering
% \end{figure}

\printbibliography

\end{document}