\documentclass[11pt]{article}

\usepackage[top=0.5in, bottom=0.5in, left=0.5in, right=0.5in]{geometry}
\usepackage{authblk}
\usepackage{hyperref}
\usepackage[utf8]{inputenc}
\usepackage{amsmath}
\usepackage{amsfonts}
\usepackage{amssymb}
\usepackage{siunitx}
\usepackage{graphicx}
\usepackage{subcaption}
\usepackage{float}
\usepackage[nottoc,numbib]{tocbibind}
\usepackage{biblatex}

\bibliography{references.bib}

\newcommand{\email}[1]{\texttt{\href{mailto:#1}{#1}}}

\title{CMSC 411 Project Documentation}
\author{Robert Rose, Alex Flaerty and Mina Beshai}

\makeatletter
\let\inserttitle\@title
\let\insertauthor\@author
\makeatother

\begin{document}

\begin{center}
  \LARGE{\inserttitle}

  \Large{\insertauthor}
\end{center}

\section{Approach}

In the beginning we were hoping to implement the algorithms entirely on our own, but unfortunately
the resources out there on CORDIC, especially beyond simply sine and cosine, are rather sparse.
Although the original CORDIC paper is somewhat helpful, we didn't initially go looking for
it.\cite{volder1959cordic}\\

When we first went to research sine and cosine approximation, we mostly got resources detailing Taylor
series and approximation using multiplication.\cite{coranac} It wasn't until class on Thursday when
Professor Cain sent out code he found and talked about the project in class that we knew what to
look for.\cite{cainemail} After some digging on GitHub, Mina found an old project from last year that
we could tear apart and re-document.\cite{oldproj} We then went through and improved the documentation
for that project and then looked for any areas of improvement. We did actually manage to find one area
where the code could potentially be improved. In addressing the lookup table, we can potentially shave
off one instruction by using ARM's register offset addressing instead of adding to the address register
and then referencing it.\cite{registeroffsetmanual}\\

We also attempted the extra credit, since partial credit is awarded. One of the sources we had provided
C code for a $log()$ estimator but we chose not to implement it in ARM because we weren't sure if $log()$
was available for extra credit, only $ln()$ is listed on the handout.\cite{efunclog} After submitting the
main project as well we also tried to write $sqrt()$ in ARM as well and succeeded, although with using
multiply in order to simplify the algorithm. We found C code online and then used the sine-cosine ARM code
as a base in order to write it in ARM.\cite{sqrtcordic}

\section{Results}

In order to get the CPI and other stats, we first needed to run ARMSim multiple times in order to get
the average number of instructions per second from the console output. We borrowed this portion of the
procedure from a previous group's project that we found on GitHub.\cite{oldreport} For the sine and
cosine script we ran eight trials but one of the trials resulted in an extremely large instructions
per second, so it was discarded as an outlier. When we reran that run with the same inputs, the result
was equally as large. We're not sure if this is an error in the program or simply something that happens
for some values.\\

\begin{equation}
    IPS_{sincos} = \frac{4739 + 4839 + 4675 + 4829 + 4926 + 4387 + 4939}{7} = 4762
\end{equation}

Unfortunately, as we're working on the report now we realized that the previous report we found was a
piece of shit and wasn't actually correct.\cite{oldreport} We don't actually know how we're supposed to
calculate the CPI without manually stepping through the program, something not really possible since it
has hundreds of instructions total.

% we can use this as a base for a table containing execution times.
% \begin{table}[h]
%  \centering
%  \caption{Cluster and Centroid Simple Model Results}
%  \begin{tabular}{||c c c c||}
%  \hline
%  Index & Name & Score & Execution Time \\ [0.5ex]
%  \hline\hline
%  4 & kmeans\_4\_centroids & 0.089267 &    47.68s \\
%  \hline
%  5 & kmeans\_5\_clusters & 0.089349 & 47.73s \\
%  \hline
%  1 & kmeans\_3\_clusters & 0.089359 & 46.66s \\
%  \hline
%  6 & kmeans\_5\_centroids & 0.089408 &    48.28s \\
%  \hline
%  3 & kmeans\_4\_clusters & 0.089562 & 48.72s \\
%  \hline
%  2 & kmeans\_3\_centroids & 0.089677 & 47.29s \\
%  \hline
%  0 & original &   0.092813 & 28.97s \\
%  \hline
% \end{tabular}
% \end{table}

% \begin{figure}[h]
% \caption{K-Means with Five Clusters}
% \includegraphics[width=0.6\textwidth]{clusters.png}
% \centering
% \end{figure}

\printbibliography

\end{document}